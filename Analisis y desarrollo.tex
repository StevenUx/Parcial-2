\documentclass{article}
\usepackage[utf8]{inputenc}
\usepackage[spanish]{babel}
\usepackage{listings}
\usepackage{graphicx}
\graphicspath{ {images/} }
\usepackage{cite}

\begin{document}

\begin{titlepage}
    \begin{center}
        \vspace*{1cm}
            
        \Huge
        \textbf{Informe de análisis y diseño}
            
        \vspace{0.5cm}
        \LARGE
        Parcial 2
            
        \vspace{1.5cm}
            
        \textbf{Ana María Ardila Ariza\\ Brayan Gomez Cardona}
            
        \vfill
            
        \vspace{0.8cm}
            
        \Large
        Despartamento de Ingeniería Electrónica y Telecomunicaciones\\
        Universidad de Antioquia\\
        Medellín\\
        Septiembre de 2021
            
    \end{center}
\end{titlepage}

\tableofcontents
\newpage
\section{Sección introductoria}\label{intro}
A continuación se expondrá el analisis que se le dió al parcial número 2 de la materia informática 2, en el cual daremos una idea inicial de como podríamos planear este desafío. 

\section{Sección de contenido} \label{contenido}
\subsection{Analisis}
En primer lugar se tiene que hay que hacer una aplicación a nuestra carrera con base a nuestros conocimientos adquiridos, como lo son las pantallas leds pero esta vez llevado a un nivel mayor que en el anterior proyecto, en el cual ahora tenemos que presentar una imagen completa en la pantalla de leds que nosotros configuremos, este proyecto nos genera un reto pero a la vez nos motiva a poder obtener un buen resultado final.
\subsection{Planeación inicial}
Para poder hacer un buen trabajo nos planteamos retos y nos organizaremos de la siguiente manera:\\
1. Planear el circuito hecho en Tinkercard, en primer lugar idearemos como hacer el circuito, como haremos y conectaremos la pantalla de leds, entre otros, esto para empezar a trabajar plenamente en el codigo.\\
2. Hacer el codigo, esto tomará un poco mas de tiempo, se planea hacerlo en QT para luego llevarlo a Tinkercard, se harán varios borradores hasta llegar al resultado mas deseado final, el cual sea una implementación exitosa.\\
3. Llevar el codigo hecho a Tinkercard para su acoplamiento con su nuevo medio, este aunque nos genera un reto, el resultado que se espera es un acoplamiento exitoso con un codigo limpio y facil de leer.\\
4. Hacer que la interacción del usuario sea muy comoda, el usuario es al que va dirijido nuestro proyecto, por lo cual sería algo absurdo no pensar en su comodidad a la hora de implementar el programa, por eso nuestro objetivo es el de poder hacer sentir al usuario feliz y con una buena impresión del programa.\\
5. Ejecutar el programa, esta parte es la final en la cual se espera que el programa actue de manera exitosa, cumpliendo con los objetivos planteados en el proyecto y nuestros objetivos personales propuestos en este documento.\\


\end{document}